\documentclass[fleqn,usenatbib]{mnras}

% MNRAS is set in Times font. If you don't have this installed (most LaTeX
% installations will be fine) or prefer the old Computer Modern fonts, comment
% out the following line
% \usepackage{newtxtext,newtxmath}
% Depending on your LaTeX fonts installation, you might get better results with one of these:
% \usepackage{mathptmx}
% \usepackage{txfonts}

% Use vector fonts, so it zooms properly in on-screen viewing software
% Don't change these lines unless you know what you are doing
\usepackage[T1]{fontenc}

% Allow "Thomas van Noord" and "Simon de Laguarde" and alike to be sorted by "N" and "L" etc. in the bibliography.
% Write the name in the bibliography as "\VAN{Noord}{Van}{van} Noord, Thomas"
\DeclareRobustCommand{\VAN}[3]{#2}
\let\VANthebibliography\thebibliography
\def\thebibliography{\DeclareRobustCommand{\VAN}[3]{##3}\VANthebibliography}


%%%%% AUTHORS - PLACE YOUR OWN PACKAGES HERE %%%%%

% Only include extra packages if you really need them. Common packages are:
\usepackage{graphicx}
\graphicspath{ {C:/dev/spin_down/final_report/report_images/} }	% Including figure files
\usepackage{amsmath}	% Advanced maths commands
\usepackage{amssymb}	% Extra maths symbols

%%%%%%%%%%%%%%%%%%%%%%%%%%%%%%%%%%%%%%%%%%%%%%%%%%

%%%%% AUTHORS - PLACE YOUR OWN COMMANDS HERE %%%%%

% Please keep new commands to a minimum, and use \newcommand not \def to avoid
% overwriting existing commands. Example:
\newcommand{\pcm}{\,cm$^{-2}$}	% per cm-squared

%%%%%%%%%%%%%%%%%%%%%%%%%%%%%%%%%%%%%%%%%%%%%%%%%%

%%%%%%%%%%%%%%%%%%% TITLE PAGE %%%%%%%%%%%%%%%%%%%

% Title of the paper, and the short title which is used in the headers.
% Keep the title short and informative.
% \title[Short title, max. 45 characters]{MNRAS \LaTeXe\ template --}
\title{PHYS6013 Midterm Report: A Machine Learning Approach to Cool Star Spin-Down}

% The list of authors, and the short list which is used in the headers.
% If you need two or more lines of authors, add an extra line using \newauthor
\author[E. Doodson et al.]{
Ely Doodson,$^{1,2,3}$\thanks{E-mail: ely.doodson@cfa.harvard.edu}
Cecilia Garraffo,$^{2,3}$
Pavlos Protopapas$^{3}$
and Jeremy J. Drake$^{2}$
\\
% List of institutions
$^{1}$School of Physics and Astronomy, University of Southampton,
Southampton, SO17 1BJ, United Kingdom\\
$^{2}$Harvard-Smithsonian Center for Astrophysics, 60 Garden St, Cambridge, MA 02138, United States \\
$^{3}$Institute for Applied Computational Science, Harvard University, Cambridge, MA 02138, United States
}

\pubyear{2020}

%imported for strikethrough
\usepackage{ulem}
\usepackage{float}

\begin{document}
\label{firstpage}
\pagerange{\pageref{firstpage}--\pageref{lastpage}}
\maketitle

% Abstract of the paper
\begin{abstract}
	Observations of young open clusters have shown a bimodal distribution in the rotation
	periods of cool stars.
	This bi-modality stems from stars having fast or slow rotation periods.
	The evolution of this trend through time suggests a fast transition from fast to slow rotating.
	Our current understanding of cool star spin down, through magnetic braking, accounts for the slow rotators branch, while the fast rotators  remain somewhat of a mystery.

	Our goal is to build a predictive probabilistic spin-down model that links the period of a star at any given mass and age.
	We use machine learning to predict the age at which each star transitions from fast to slow-rotation.
	Using a graphical model we will translate the distribution of initial periods into a rotation period probability distribution for a given mass and age.
\end{abstract}

%%%%%%%%%%%%%%%%%%%%%%%%%%%%%%%%%%%%%%%%%%%%%%%%%%
%%%%%500wordsInTotalBeforeBody%%%%%%%%%%%%%%%
%%%%%%%%%%%%%%%%% BODY OF PAPER %%%%%%%%%%%%%%%%%%

\section{Introduction}
\begin{itemize}

	\item Establish the context of the work being reported.
	      \begin{itemize}
		      \item Lay out the physics of the problem
		      \item Discuss relevant primary research literature (with citations)
		      \item Summarize current understanding of the problem
	      \end{itemize}

	\item State the purpose of the work
	      \begin{itemize}
		      \item Hypothesis/
		      \item Question/
		      \item Or problem you investigated
	      \end{itemize}

	\item Explain your rationale and approach and, whenever possible, the possible outcomes your study can reveal.

\end{itemize}

\section{Data Collection}
\subsubsection{Data Trimming}
\subsubsection{MIST Tables}
\subsubsection{Interpolation of Results}
\subsubsection{Clusters - Displayed}
\subsubsection{Clusters - Combined}


\section{Preliminary Methods}
\subsection{Unsupervised Clustering}
\subsection{Polynomial Ridge Regression}

\section{Gaussian Mixture Models}
Clustering is an unsupervised learning problem where we intend to find clusters of points in our dataset that share some common characteristics.

One important characteristic of K-means is that it is a hard clustering method, which means that it will associate each point to one and only one cluster.
A limitation to this approach is that there is no uncertainty measure or probability that tells us how much a data point is associated with a specific cluster.

A Gaussian Mixture is a function that is comprised of several Gaussians, each identified by $k \in \{1,\dots, K\}$, where $K$ is the number of clusters of our dataset. Each Gaussian $k$ in the mixture is comprised of the following parameters:

\begin{equation}
	\label{eq:gaussian_density_function}


\end{equation}
\section{Final Model Analysis}
\subsubsection{}
\subsection{The Importance of Initial Period, $P_i$} \label{sec:initial_period}

%%%%%%%%%%%%%%%%%%%%%%%%%%%%%%%%%%%%%%%%%%%%%%%%%%

%%%%%%%%%%%%%%%%%%%%REFERENCES%%%%%%%%%%%%%%%%%%

%The best way to enter references is to use BibTeX:

\bibliographystyle{mnras}
\bibliography{references}

%%%%%%%%%%%%%%%%%APPENDICES%%%%%%%%%%%%%%%%%%%%%

% \appendix

% \section{Some extra material}

% If you want to present additional material which would interrupt the flow of the main paper,
% it can be placed in an Appendix which appears after the list of references.

%%%%%%%%%%%%%%%%%%%%%%%%%%%%%%%%%%%%%%%%%%%%%%%%%%


% Don't change these lines
\bsp	% typesetting comment
\label{lastpage}
\end{document}

% End of mnras_template.tex